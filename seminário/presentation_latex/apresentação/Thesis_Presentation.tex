%%%%%%%%%%%%%%%%%%%%%%%%%%%%%%%%%%%%%%%%%%%%%%%%
%%%%%%%%%%%%%%%%%%%%%%%%%%%%%%%%%%%%%%%%%%%%%%%%%%
%%
%% Based one the "beamer-greek-two" template provided 
%% by the Laboratory of Computational Mathematics, 
%% Mathematical Software and Digital Typography, 
%% Department of Mathematics, University of the Aegean
%% (http://myria.math.aegean.gr/labs/dt/)
%%
%% Adapted by John Liaperdos, October-November 2014
%% (ioannis.liaperdos@gmail.com)
%%
%% Last update: 22/06/2017 (English Support)
%%
%%%%%%%%%%%%%%%%%%%%%%%%%%%%%%%%%%%%%%%%%%%%%%%%%%
%%%%%%%%%%%%%%%%%%%%%%%%%%%%%%%%%%%%%%%%%%%%%%%%%%
%%
\PassOptionsToPackage{unicode}{hyperref}
\PassOptionsToPackage{naturalnames}{hyperref}
\documentclass{beamer} 
%\usepackage{babel}
%\usepackage[utf8]{inputenc}


%%% FONT SELECTION %%%%%%%%%%%%%%%%%
%%% we choose a sans font %%%%%%%%%%
\usepackage{kmath,kerkis} 
%\usepackage[default]{gfsneohellenic} 
%%%%%%%%%%%%%%%%%%%%%%%%%%%%%%%%%%%%

\usepackage{color}
\usepackage{amsmath}
\usepackage{amssymb}
\usepackage[T1]{fontenc}
\usepackage[utf8]{inputenc}
\usepackage[brazil]{babel}
\usepackage{epstopdf}
\usepackage{graphicx}
\graphicspath{{./images/}}
\usepackage{tikz}
\usepackage{enumitem}
\usepackage{lmodern}
\usepackage{forloop}
\usepackage{calc}    % we need this to be able to multiply (*)
\usetikzlibrary{lindenmayersystems}
\usepackage{mathtools}
\usepackage{pgfplots}
\usepackage{hyperref}
\usetikzlibrary{math}
\pgfplotsset{compat=1.18} 


\hypersetup{
  colorlinks=true,
  linkcolor=blue,
  urlcolor=blue
}

%%
% load TEI-Pel - specific layout
\usepackage{TeiPel_En_Beamer_Layout}
\setTeipelLayout{draft,newlogo}% options: "draft", "newlogo"

\makeatletter
\def\moverlay{\mathpalette\mov@rlay}
\def\mov@rlay#1#2{\leavevmode\vtop{%
   \baselineskip\z@skip \lineskiplimit-\maxdimen
   \ialign{\hfil$\m@th#1##$\hfil\cr#2\crcr}}}
\newcommand{\charfusion}[3][\mathord]{
    #1{\ifx#1\mathop\vphantom{#2}\fi
        \mathpalette\mov@rlay{#2\cr#3}
      }
    \ifx#1\mathop\expandafter\displaylimits\fi}
\makeatother

\newcommand{\cupdot}{\charfusion[\mathbin]{\cup}{\cdot}}
\newcommand{\bigcupdot}{\charfusion[\mathop]{\bigcup}{\cdot}}

%%%%%%%%%%%%%%%%%%%%%%%%%%%%%%%%%%%%%%%%%%%%%%%%%%%%%%%%%%%%
% Thesis Info %%%%%%%%%%%%%%%%%%%%%%%%%%%%%%%%%%%%%%%%%%%%%%
%%%%%%%%%%%%%%%%%%%%%%%%%%%%%%%%%%%%%%%%%%%%%%%%%%%%%%%%%%%%
	% title
		\title{Sistema de Rossler}
		\subtitle{Tópicos de matemática computacional}	
	% author 
    % (In the mandatory argument "{}", separate multiple
    % authors with "\and" - use "\\" for better author name formatting
    % in the title page. In the optional argument "[]" include all
	% author names, with no "\and" or text formatting macros.)
	% Example: 
    %\author[A. Author Albert Einstein]{Anthony Author \and Albert Einstein}
		\author[]{Kaique M. M. de Oliveira \\ Caio U. Martins}
	% supervisor	
                \supervisor{Professor: Luciano Aparecido Magrini}
		%\supervisor{Orientador}{Prof. Dr.}{Luís Antônio C. dos Santos}
	% date
		\presentationDate{24 de Fevereiro de 2024}
%%%%%%%%%%%%%%%%

\begin{document}

% typeset front slides
	\typesetFrontSlides

%%%%%%%%%%%%%%%%
% Your Slides Start here:
%%%%

%SECTION-----------------------------------------

\section{Introdução}


\begin{frame}{Introdução}

    \begin{exampleblock}{O que está por vir?}
    	\begin{itemize}
    	\item [$\bullet$] Trazer uma abordagem histórica da Teoria do Caos
		\item [$\bullet$] Explicar quem foi Otto Rossler
		\item [$\bullet$] Apresentar um estudo matemático-computacional do Sistema de 	Rossler 
		\end{itemize}
    \end{exampleblock}
    
        \begin{exampleblock}{Objetivo}
    	\begin{itemize}
		\item [$\bullet$] Nosso objetivo do seminário é explicar um pouco sobre a história de Otto Rossler contextualizada nos acontecimentos na teoria de Sistemas Dinâmicos como foco na Teoria do Caos.
    	\end{itemize}
    \end{exampleblock}
    
\end{frame}


%%%%%%%%%%%%%%%%%%%%%%%%%%%%%%%%%%%%%%%%%%%%%%%%%%%%%%%%%%%%%%%%%%%%%%%%%%%%%%%%%%%%%%%%%%%%%%%%%

\section{Pesquisa histórica}
\subsection{Contextualização da Teoria do Caos}
\begin{frame}{O contexto}
    \begin{exampleblock}{Teoria do Caos}
        \begin{itemize}
		\item [$\bullet$] A teoria do caos é uma teoria matemática, que permite a descrição de fenômenos relacionados a sistemas dinâmicos.
		\item [$\bullet$] Um sistema dinâmico é um sistema que muda com o tempo devido a uma causa e um efeito.
		\item [$\bullet$] Um evento caótico é um evento que por fins práticos é impossível de se prever o seu desenvolvimento conforme o tempo aumenta.
        \end{itemize}
    \end{exampleblock}
    \begin{exampleblock}{Newton e a causalidade}
    	\begin{itemize}
    		\item [$\bullet$] Uma das primeiras concepções sobre os sistemas dinâmicos é o princípio da causalidade, que é a propriedade de um evento futuro ser  unicamente determinado pelas propriedades do presente.
    	\end{itemize}
    \end{exampleblock}
\end{frame}

\begin{frame}{Determinismo}
			
	\begin{exampleblock}{Laplace e o determinismo}
		\begin{itemize}
			\item [$\bullet$] O conceito de determinismo se transformou na discussão presente no livro "Le système de la nature" de 1770, na qual o filosofo d'Holbach faz uma afirmação sobre a viabilidade de calcular os efeitos de uma determinada causa de modo universal.
			
			\item [$\bullet$] Mas, é Laplace que clarificou o conceito do que é determinismo universal, que diz que o universo é unicamente determinado pelas leis da física. "O universo no bater do relógio".
	
		\end{itemize}
	\end{exampleblock}
\end{frame}

\begin{frame}{Sistemas dinâmicos}
	\begin{exampleblock}{Dinâmica estátistica}
		\begin{itemize}
			\item [$\bullet$] Poincaré e o espaço de fase
				\\ - Representação de um espaço abstrato, no qual se aplica certas leis físicas com uma certa série de parâmetros.
			\item [$\bullet$] Kolmogorov e o sistemas dinâmicos
				\\ - Modelos lineares e modelos não lineares.
				\\ - A soma das causas pode não necessariamente ser a soma dos efeitos.
		\end{itemize}
	\end{exampleblock}
	
	\begin{exampleblock}{Lorenz e o efeito borboleta}
	\begin{itemize}
		\item [$\bullet$] \textit{"Predictability: does the flap of a butterfly's wing in Brazil set off a tornado in Texas?}"
		\item [$\bullet$] Pequenas variações no estado inicial podem induzir magnitudes de ordens muito maiores do estado final.
	\end{itemize}
	\end{exampleblock}
	\end{frame}

%%%%%%%%%%%%%%%%%%%%%%%%%%%%%%%%%%%%%%%%%%%%%%%%%%%%%%%%%%%%%%%%%%%%%%%%%%%%%%%%%%%%%%%%%%%%%%
	
\subsection{A história de Otto Rossler}
\begin{frame}{O que Otto Rossler tem haver com isso?}
	
	\begin{exampleblock}{Breve Biografia}
		\begin{itemize}
		\item [$\bullet$] Otto Rossler nasceu na Alemanhã e foi um bioquímico conhecido pela equação teórica do Sistema de Rossler. 
		Escreveu mais de 300 artigos científicos e estudou medicina na Universidade de Tuebingen.
		\item [$\bullet$] Possui uma grande fase da sua vida investigando a resoluções de equações diferenciais da bioquímica, usando computadores eletrônicos e digitais da época.
		
		\item [$\bullet$] No começo de 1970, Otto Rossler fez seus primeiros contatos com Art Winfree, que trocavam cartas sobre sistemas dinâmicos.
		 
		\end{itemize}
	\end{exampleblock}
\end{frame}

\begin{frame}{O que Otto Rossler tem haver com isso?}
	
	\begin{exampleblock}{Breve Biografia}
		\begin{itemize}
			\item [$\bullet$] Em 1975, nas trocas de carta entre Otto e Winfree, Art desafiou Rossler a encontrar uma reação bioquímica que reproduzia o atrator de Lorenz e enviou um conjunto de 10 papers de seus arquivos para ele.
			
			\item [$\bullet$] Nesse conjunto um dos papers era o de Lorenz, no qual Otto ficou bastante impressionado.
			
			\item [$\bullet$] Muito influenciado, Otto falhou em encontrar a tal reação, mas encontrou um atrator mais simples, no qual deu origem a seu primeiro paper sobre o sistema de Rossler.
		\end{itemize}
	\end{exampleblock}
\end{frame}

%%%%%%%%%%%%%%%%%%%%%%%%%%%%%%%%%%%%%%%%%%%%%%%%%%%%%%%%%%%%%%%%%%%%%%%%%%%%%%%%%%%%%%%%%%%%%%%%%
\section{Sistema de Rossler}
\subsection{Abordagem Analítica}
\begin{frame}{O sistema de Rossler}
	\begin{exampleblock}
		<1->{O Definição}
		O sistema de Rossler é definido pelo seguinte sistema
		\begin{equation}
			\begin{cases}
				\frac{dx}{dt} &= -y - z \\
				\frac{dy}{dt} &= x + ay \\
				\frac{dz}{dt} &= b + z(x - c)
			\end{cases}
		\end{equation}
		onde $a, b, c \in \mathbb{R}$
	\end{exampleblock}

	\begin{exampleblock}
		<2-> {Pontos de equilíbrio} 
		Note que o sistema de Rossler é uma equação diferencial autônoma. Podemos
		encontrar os seus de equilíbrio fazendo $f(x) = 0$.
	\end{exampleblock}
\end{frame}

\begin{frame}
	\begin{exampleblock}
		<1-> {Pontos de equilíbrio} 
		Resolvendo a equação, obtemos que o sistema de Rossler apresenta dois pontos de equilíbrio

		\noindent
		\begin{equation}
			\resizebox{0.9 \hsize}{!}{
			$(x_{\pm}, y_{\pm}, z_{\pm})	=
			\left( \frac{c\pm\sqrt{c^2-4ab}}{2},
			-\left(\frac{c\pm\sqrt{c^2-4ab}}{2a}\right),
			\frac{c\pm\sqrt{c^2-4ab}}{2a}\right)$
		}
		\label{eq:equilibrium}
		\end{equation}
		
	\end{exampleblock}

	\begin{exampleblock}
		<2-> {O jacobiano } 
		Podemos calcular o Jacobiano do sistema para investigar os pontos de equilíbrio

		\begin{equation}
J = \begin{pmatrix}
-\frac{\partial f_1}{\partial x} & -\frac{\partial f_1}{\partial y} & -\frac{\partial f_1}{\partial z} \\
-\frac{\partial f_2}{\partial x} & -\frac{\partial f_2}{\partial y} & -\frac{\partial f_2}{\partial z} \\
-\frac{\partial f_3}{\partial x} & -\frac{\partial f_3}{\partial y} & -\frac{\partial f_3}{\partial z}
\end{pmatrix} = 
\begin{pmatrix}0 & -1 & -1 \\ 1 & a & 0 \\ z & 0 & x-c\\\end{pmatrix}
		\label{jacobian}
		\end{equation}
		
	\end{exampleblock}
\end{frame}

%%%%%%%%%%%%%%%%%%%%%%%%%%%%%%%%%%%%%%%%%%%%%%%%%%%%%%%%%%%%%%%%%%%%%%%%%%%%%

\begin{frame}
		
	\begin{exampleblock}
		<1-> {Autovalores de $J$} 
		Podemos encontrar os autovalores da matriz $J$ resolvendo a seguinte cúbica.

		\begin{equation}
			-\lambda^3+\lambda^2(a+x-c) + \lambda(ac-ax-1-z)+x-c+az =0
			\label{eqlamb}
		\end{equation}
	
		Ao aplicarmos os pontos de equilíbrio \ref{eq:equilibrium} a equação \ref{eqlamb}, podemos 
		obter alguns resultados analíticos utilizando do seguinte teorema
	\end{exampleblock}
	
	\begin{exampleblock}
		<2-> {Teorema}
Um ponto de equilíbrio $x*$ da equação diferencial $x' = f(x)$ é estável se todos os autovalores de $J*$, 
o Jacobiano avaliado em $x*$, tiverem partes reais negativas. Um ponto de equilíbrio é instável se 
pelo menos um dos seus autovalores tiver parte real positiva.
	\end{exampleblock}

\end{frame}


%%%%%%%%%%%%%%%%%%%%%%%%%%%%%%%%%%%%%%%%%%%%%%%%%%%%%%%%%%%%%%%%%%%%%%%%%%%%%%%%%%%%%%%%%%%%%%

\section{Análise numérico-computacional}

\begin{frame}

	\begin{exampleblock}
		<1-> {Aplicações do teorema}
		Um ponto de equilíbrio é estável se para toda condição inicial suficientemente
		perta do equilíbrio, as trajetórias da solução permanecem arbitrariamente pertas
		do ponto de equilíbrio. Um ponto de equilíbrio instável é o contrário, as 
		trajetórias são repelidas.
	\end{exampleblock}

	
	\begin{exampleblock}
		<2-> {Abordagem computacional}
		Utilizaremos uma abordagem mais experimental e computacional para analizarmos 
		os impactos dos dos diferentes valores de $a, b, c$ e das diferentes contições 
		iniciais. 
	\end{exampleblock}
	
\end{frame}


%%%%%%%%%%%%%%%%%%%%%%%%%%%%%%%%%%%%%%%%%%%%%%%%%%%%%%%%%%%%%%%%%%%%%%%%%%%%%%%%%%%%%%%%%%%%%%

\begin{frame}
	\begin{exampleblock}
		<1-> {Valores base para c}
			Para $a, b = 0.1$ teremos que 
			\begin{itemize}
				\item[$\bullet$] c = 4, órbita com período 1
				\item[$\bullet$] c = 6, órbita com período 2
				\item[$\bullet$] c = 8.5, órbita com período 2
				\item[$\bullet$] c = 9, órbita caótica
				\item[$\bullet$] c = 12, órbita com período 3
				\item[$\bullet$] c =18, órbita caótica
			\end{itemize}	
	\end{exampleblock}
\end{frame}


%%%%%%%%%%%%%%%%%%%%%%%%%%%%%%%%%%%%%%%%%%%%%%%%%%%%%%%%%%%%%%%%%%%%%%%%%%%%%%%%%%%%%%%%%%%%%%
\begin{frame}
	\begin{exampleblock}
		<1-> {Valores base para a}
			Para $b = 0.1$ e $c = 4$ teremos que 
			\begin{itemize}
				\item[$\bullet$] a = 0.05, 0.1 órbitaa com período 1
				\item[$\bullet$] a = 0.15, órbita com período 2
				\item[$\bullet$] a = 0.2, órbita com período 4
				\item[$\bullet$] a = 025, órbita com período 2
				\item[$\bullet$] a = 0.28, órbita com período 4
				\item[$\bullet$] a = 0.3, órbita caótica
				\item[$\bullet$] a = 0.33, órbita com período 3
				\item[$\bullet$] a = 0.35, órbita com caótica
			\end{itemize}	
	\end{exampleblock}
\end{frame}

%%%%%%%%%%%%%%%%%%%%%%%%%%%%%%%%%%%%%%%%%%%%%%%%%%%%%%%%%%%%%%%%%%%%%%%%%%%%%%%

\begin{frame}
	\begin{exampleblock}
		<1-> {Valores base para b}
			Para $a = 0.1$ e $c = 4$ teremos que 
			\begin{itemize}
				\item[$\bullet$] b = 0.01, órbita com período 2
				\item[$\bullet$] $b \in [0.04, 1]$ órbita com período 1
			\end{itemize}	
	\end{exampleblock}

	\begin{exampleblock}
		<2-> {Valores base para b}
			Para $a = 0.1$ e $c = 7.3$ teremos que 
			\begin{itemize}
				\item[$\bullet$] b = 0.04, órbita com período 4
				\item[$\bullet$] b = 0.1, 0.15, 0.25, 0.3 órbita com período 2
				\item[$\bullet$] $b \in [0.04, 1]$ órbita com período 1
			\end{itemize}	
	\end{exampleblock}
\end{frame}

%%%%%%%%%%%%%%%%%%%%%%%%%%%%%%%%%%%%%%%%%%%%%%%%%%%%%%%%%%%%%%%%%%%%%%%%%%%%%%%%%%%%%%%%%%%%%%
\section{Referências}
\begin{frame}{Referências}
	\begin{thebibliography}{3}
		\bibitem{PPMID: 17969865; PMCID: PMC3202497.}
		Oestreicher C.,
		\newblock A history of chaos theory.
		\newblock\emph{PMC PubMed Central},
		\oldstylenums{2007}.
		
		\beamertemplatearticlebibitems
		\bibitem{PPMID: 17969865; PMCID: PMC3202497.}
		STROGATZ, Steven H. 
		\newblock Nonlinear dynamics and chaos with student solutions manual: With applications to physics, biology, chemistry, and engineering.
		\newblock\emph{CRC press},
		\oldstylenums{2018}.
     \end{thebibliography}
\end{frame}

\begin{frame}{Referências}
	\begin{thebibliography}{3}
		
		\beamertemplatearticlebibitems
		\bibitem{}
		ATOMOSYD
		\newblock OTTO E. RÖSSLER
		\newblock\emph{http://www.atomosyd.net/spip.php?article6},
		\oldstylenums{2008}.
		
		\beamertemplatearticlebibitems
		\bibitem{International Journal of Bifurcation and ChaosVol. 20, No. 11}
		INFLUENCES ON OTTO E. ROSSLER’S EARLIEST PAPER ON CHAOS
		\newblock C. LETELLIER and V. MESSAGER
		\newblock\emph{International Journal of Bifurcation and Chaos Vol. 20, No. 11},
		\oldstylenums{2010}.
		
	\end{thebibliography}
\end{frame}

\begin{frame}{Referências}
    \begin{thebibliography}{5} % Adjust the number if needed
		
		
		\beamertemplatearticlebibitems
		\bibitem{Doering-Lopes}
		Equações Diferenciais Ordinárias
		\newblock Claus Ivo Doering e Artur Oscar Lopes
		\newblock \emph{IMPA, 2016}.
		\newblock \emph{ISBN: 978-85-244-0425-2}, 6ª edição.

        \beamertemplatearticlebibitems
		\bibitem{Viana-Espinar}
		Differential Equations: A Dynamical Systems Approach to Theory and Practice
		\newblock Marcelo Viana, José M. Espinar
		\newblock \emph{Graduate Studies in Mathematics, 2021}.
		
	\end{thebibliography}
\end{frame}
%%%%%%%%%%%%%%%%%%%%%%%%%%%%%%%%%%%%%%%%%%%%%%%%%%%%%%%%%%%%%%%%%%%%%%%%%%%%%%%%%%%%%%%%%%%
\end{document}

