
\documentclass[12pt,a4paper]{article} % Fonte 12, Papel A4
\usepackage[brazil]{babel} % Hifenizac¸˜ao em portuguˆes
\usepackage{amsmath}
\usepackage{amsthm}
\usepackage{exscale}
\usepackage{graphicx}
\usepackage{amsfonts}
\usepackage{amstext}
\usepackage{amssymb}
\usepackage{tabto}
\usepackage{mathrsfs}
\usepackage[top=2cm,bottom=3cm,left=3cm,right=2cm]{geometry}
\usepackage[utf8]{inputenc}
\usepackage{enumerate}
\usepackage{tikz}
\usepackage{pgfplots}
\pgfplotsset{compat = newest}
\usepgfplotslibrary{colormaps}

\def\modulus{sqrt(1+(1/(x^2 - 1))^2)}

\newtheorem{teo}{Teorema}
\newtheorem{lema}[teo]{Lema}
\newtheorem{cor}[teo]{Corolário}
\newtheorem{prop}{Proposição}
\newtheorem{obs}{Observação}
\theoremstyle{definition}
\newtheorem{defi}{Definição}
\newtheorem{exem}{Exemplo}
\newtheorem{ex}{Exercício}
\numberwithin{equation}{ex}
\newenvironment{sol}
{\renewcommand\qedsymbol{$\blacksquare$}\begin{proof}[Solução]}
  {\end{proof}}

\newenvironment{costumteo}[1]
  {\renewcommand\theteo{#1}\teo}
  {\endteo}
  
\newenvironment{costumlema}[1]
  {\renewcommand\thelema{#1}\lema}
  {\endlema}
  
\newenvironment{costumdefi}[1]
  {\renewcommand\thedefi{#1}\defi}
  {\enddefi}

\newenvironment{costumprop}[1]
  {\renewcommand\theprop{#1}\prop}
  {\endprop}
\newenvironment{costumex}[1]
  {\renewcommand\theex{#1}\ex}
  {\endex}

\theoremstyle{definition}
\newtheorem{definition}{Definição}[section]


\renewcommand{\qedsymbol}{$\blacksquare$}
\newcommand{\R}{\mathbb{R}}
\newcommand{\Q}{\mathbb{Q}}
\newcommand{\N}{\mathbb{N}}
\renewcommand{\thesection}{\Roman{section}}  
\renewcommand{\thesubsection}{\thesection.\Roman{subsection}}
\title{Seminário: Tópicos de matemática computacional \\ Sistema de Rossler}
\author{Kaique M. M. de Oliveira \\ Caio U. Martins}

\begin{document}

\maketitle
\date{}

\section{Pesquisa histórica}

\subsection{Contextualização da Teoria do Caos}

\begin{itemize}
	\item [$\bullet$] Quais são os tópicos discutidos
	
	- A teoria do caos é uma teoria matemática, que permite a descrição de fenômenos relacionados a sistemas dinâmicos.

	- Um sistema dinâmico é um sistema que muda com o tempo devido a uma causa e um efeito.

	- Um evento caótico é um evento que por fins práticos é impossível de se prever o seu desenvolvimento conforme o tempo aumenta.
	
	\item [$\bullet$] Newton e a causalidade
	
	Uma das primeiras concepções sobre os sistemas dinâmicos é o princípio da causalidade, que é a propriedade de um evento futuro ser  unicamente determinado pelas propriedades do presente.
	
	\item [$\bullet$]  Laplace e o determinismo
	
	- O conceito de determinismo se transformou na discussão presente no livro "Le système de la nature" de 1770, na qual o filosofo d'Holbach faz uma afirmação sobre a viabilidade de calcular os efeitos de uma determinada causa de modo universal.
	
	- Mas, é Laplace que clarificou o conceito do que é determinismo universal, que diz que o universo é unicamente determinado pelas leis da física. "O universo no bater do relógio".
	
	\item [$\bullet$] Poincaré e o espaço de fase
	
	- Representação de um espaço abstrato, no qual se aplica certas leis físicas com uma certa série de parâmetros.
		
	\item [$\bullet$] Kolmogorov e o sistemas dinâmicos
	
	- Modelos lineares e modelos não lineares.
	
	- A soma das causas pode não necessariamente ser a soma dos efeitos.
		
	\item [$\bullet$] Lorenz e o efeito borboleta
		
	- "Predictability: does the flap of a butterfly's wing in Brazil set off a tornado in Texas?
		
	- Pequenas variações no estado inicial podem induzir magnitudes de ordens muito maiores do estado final.
	
\end{itemize}

\subsection{Breve Biografia de Otto Rossler e os sistemas dinâmicos}

	 Otto Rossler nasceu na Alemanhã e foi um bioquímico conhecido pela equação teórica do Sistema de Rossler. 
	
	Escreveu mais de 300 artigos científicos e estudou medicina na Universidade de Tuebingen.
	
	Possui uma grande fase da sua vida investigando a resoluções de equações diferenciais da bioquímica, usando computadores eletrônicos e digitais da época.
			
 	No começo de 1970, Otto Rossler fez seus primeiros contatos com Art Winfree, que trocavam cartas sobre sistemas dinâmicos.

	Em 1975, nas trocas de carta entre Otto e Winfree, Art desafiou Rossler a encontrar uma reação bioquímica que reproduzia o atrator de Lorenz e enviou um conjunto de 10 papers de seus arquivos para ele.
			
	Nesse conjunto um dos papers era o de Lorenz, no qual Otto ficou bastante impressionado.
			
	Muito influenciado, Otto falhou em encontrar a tal reação, mas encontrou um atrator mais simples, no qual deu origem a seu primeiro paper sobre o sistema de Rossler.


\section{Estudo matemático computacional}



\section{} 
	\begin{thebibliography}{3}
		\bibitem{}
		ATOMOSYD
		\newblock OTTO E. RÖSSLER
		\newblock\emph{http://www.atomosyd.net/spip.php?article6},
		\oldstylenums{2008}.
		
		\bibitem{International Journal of Bifurcation and ChaosVol. 20, No. 11}
		INFLUENCES ON OTTO E. ROSSLER’S EARLIEST PAPER ON CHAOS
		\newblock C. LETELLIER and V. MESSAGER
		\newblock\emph{International Journal of Bifurcation and Chaos Vol. 20, No. 11},
		\oldstylenums{2010}.
		
		\bibitem{Doering-Lopes}
		Equações Diferenciais Ordinárias
		\newblock Claus Ivo Doering e Artur Oscar Lopes
		\newblock \emph{IMPA, 2016}.
		\newblock \emph{ISBN: 978-85-244-0425-2}, 6ª edição.
		
		\bibitem{Viana-Espinar}
		Differential Equations: A Dynamical Systems Approach to Theory and Practice
		\newblock Marcelo Viana, José M. Espinar
		\newblock \emph{Graduate Studies in Mathematics, 2021}.
	\end{thebibliography}
\end{document}
